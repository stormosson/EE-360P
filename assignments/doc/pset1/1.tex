\documentclass[11pt]{article}
\usepackage[utf8]{inputenc}
\usepackage[T1]{fontenc}
\usepackage{amsmath}
\usepackage{amssymb}
\usepackage{amsthm}
\usepackage{mathtools}
\usepackage{fixltx2e}
\usepackage{longtable}
\usepackage{float}
\usepackage{array}
\usepackage{multirow}
\usepackage{multicol}
\usepackage{wrapfig}
\usepackage{rotating}
\usepackage[normalem]{ulem}
\usepackage{textcomp}
\usepackage{marvosym}
\usepackage{wasysym}
\usepackage{hyperref}
\tolerance=1000
\usepackage{caption}
\usepackage{subcaption}
\usepackage[margin=0.75in]{geometry}
\usepackage{enumerate}
\usepackage{showexpl}
\usepackage{color} %red, green, blue, yellow, cyan, magenta, black, white
\definecolor{mygreen}{RGB}{28,172,0} % color values Red, Green, Blue
\definecolor{mylilas}{RGB}{170,55,241}
\newcommand{\todo}{{\LARGE \emph{\color{red}TODO}}}
\hypersetup{
  pdfkeywords={},
  pdfsubject={}}
\newcounter{problem}
\newcounter{solution}
\usepackage{beramono}
\usepackage{listings}
\usepackage[usenames,dvipsnames]{xcolor}
\usepackage[T1]{fontenc}
\usepackage{inconsolata}
\usepackage{color}
\definecolor{pblue}{rgb}{0.13,0.13,1}
\definecolor{pgreen}{rgb}{0,0.5,0}
\definecolor{pred}{rgb}{0.9,0,0}
\definecolor{pgrey}{rgb}{0.46,0.45,0.48}
\lstset{language=Java,
  showspaces=false,
  showtabs=false,
  breaklines=true,
  showstringspaces=false,
  breakatwhitespace=true,
  commentstyle=\color{pgreen},
  keywordstyle=\color{pblue},
  stringstyle=\color{pred},
  basicstyle=\ttfamily,
  moredelim=[il][\textcolor{pgrey}]{$$},
  moredelim=[is][\textcolor{pgrey}]{\%\%}{\%\%}
}
% \parindent 0in
% \parskip 1em.
\begin{document}
\title{EE 360P:\@Concurrent and Distributed Systems \\ Assignment 1}
\author{Student: Eric Crosson (email: eric.s.crosson@utexas.edu)\\
  Student: William ``Stormy'' Mauldin (email: stormymauldin@utexas.edu)}
\date{Thursday January 21, 2016}
\maketitle

This homework contains a programming part (Q1-Q2) and a theory part (Q3-Q6). The
theory part should be written or typed on a paper and submitted at the beginning
of the class. The source code (Java files) of the programming part must be
uploaded through the canvas before the end of the due date (i.e., 11:59pm in
February 2\textsuperscript{nd}). The assignment should be done in teams of two. You
should use the templates downloaded from the course github
(https://github.com/vijaygarg1/EE-360P.git). You should not change the file
names and function signatures. In addition, you should not use package for
encapsulation. Note that, we provide some test cases for your convenience in
SimpleTest class. You do not need to modify the class. Please zip and name the
source code as [EID1 EID2].zip. \\

\begin{enumerate}
\item \textbf{(20 points)} Write a Java class \texttt{PSort} that allows
  parallel sorting of an array of integers. It provides the following
  \texttt{static} method: \\

\lstinputlisting[language=Java]{../../../src/pset/one/PSort.java}

  Use QuickSort algorithm for sorting and Runnable interface for your
  implementation. In quick sort, the input array is divided into two sub-arrays,
  and then the two sub-arrays are sorted recursively using the same
  algorithm. In your implementation, you should create two new threads for
  sorting the divided
  sub-arrays. \\

  The input array A is also the output array, which means your should put the
  sorted data back in A.  The range to be sorted extends from index begin,
  inclusive, to index end, exclusive. In other words, the range to be sorted is
  empty when begin == end. To simplify the problem, you can assume that the size
  of array A does not exceed 10,000.

\item \textbf{(20 points)} Write a Java class that allows parallel search in an
  array of integers. It provides the following static method:

\lstinputlisting[language=Java]{../../../src/pset/one/PSearch.java}

  This method creates as many threads as specified by \texttt{numThreads},
  divides the array A into that many parts, and gives each thread a part of the
  array to search for x sequentially. If any thread finds \texttt{x}, then it
  returns an index \texttt{i} such that \texttt{A[i] = x}. Otherwise, the method
  returns -1. Use \emph{Callable} interface for your implementation. Note that,
  you can assume that every element of the array is unique.

\item \textbf{(15 points)} Show that any of the following modifications to
  Peterson’s algorithm makes it incorrect:
  \begin{enumerate}[a)]
  \item A process in Peterson’s algorithm sets the \emph{turn} variable to
    itself instead of setting it to the other process.
    \begin{proof}
    	\todo
    \end{proof}
  \item A process sets the \emph{turn} variable before setting the \emph{wantCS}
    variable.
        \begin{proof}
    	\todo
    \end{proof}
  \end{enumerate}

\item \textbf{(15 points)} Show that the bakery algorithm does not work in the
  absence of \emph{choosing} variables.
  	\begin{proof}
		Assume two threads exist, \texttt{Thread i} of lower priority than \texttt{Thread j}. Assume \texttt{Thread i} is in \emph{choosing} phase, and that \texttt{Thread j} enters the \emph{choosing} phase before \texttt{Thread i} exits. Without a \emph{choosing} variable, \texttt{Thread j} cannot differentiate between \texttt{Thread i} choosing a number and \texttt{Thread i} not requesting a critical section. As such, \texttt{Thread j} is able to scan past \texttt{Thread i}'s index in \emph{numbers} and therefore both threads arrive at the same Baker's number.
		
		Assume \texttt{Thread i} enters the critical section. Since \texttt{Thread j} is of higher priority than \texttt{Thread i} (i.e. $j < i$ in line 28 of Figure~2.7 of the textbook), \texttt{Thread j} also enters the critical section, breaking the principle of mutual exclusion.
	\end{proof}
\item \textbf{(15 points)} Prove that Peterson's algorithm is free from starvation.
\begin{proof}
	Claim: Peterson's algorithm with two threads, \texttt{Thread i} and \texttt{Thread j}, will induce starvation. 
	
	Assume \texttt{Thread i} needs to access the critical section. Peterson's algorithm will not bypass a thread more than once when allowing access to a critical section. \texttt{Thread i} sets \texttt{turn = j}. Assume \texttt{Thread j} executes the critical section to completion and sets \texttt{wantCS[j] = 0}. This allows \texttt{Thread i} to execute the critical section, contradicting the claim.
\end{proof}

\item \textbf{(15 points)} Peterson's algorithm uses a multi-write variable
  \texttt{turn}. Modify the algorithm to use two variables \texttt{turn0} and
  \texttt{turn1} instead of \texttt{turn} such that \emph{P\textsubscript{0}}
  does not write to \texttt{turn1} and \emph{P\textsubscript{1}} does not write
  to \texttt{turn0}.
  
\lstinputlisting[language=Java]{../src/peterson.java}

\item \textbf{(0 points)} You are one of the recently arrested prisoners. The
  warden, a deranged computer scientist, makes the following announcement:
  \begin{itemize}
  \item All prisoners may meet together today and plan a strategy, but after
    today you will be in isolated cells and have no communication with one
    another.
  \item I have setup a ``switch room'' which contains a light switch, which is
    either on or off. The switch is not connected to anything.
  \item Every now and then, I will select one prisoner at random to enter the
    ``switch room''. This prisoner may throw the switch (from on to off, or
    vice-versa), or may leave the switch unchanged. Nobody else will ever enter
    this room while the prisoner is in the room.
  \item Each prisoner will visit the ``switch room'' arbitrarily often. More
    precisely, for any $N$, eventually each of you will visit the ``switch
    room'' at least $N$ times.
  \item At any time, any of you may declare: ``we have all visited the 'switch
    room' at least once.''  If the claim is correct, I will set you free. If the
    claim is incorrect, I will feed all of you to the crocodiles. Choose wisely!
  \end{itemize}

  Devise a winning strategy when you know that the initial state of the switch
  is off. (Hint: not all prisoners need to do the same thing.)

\end{enumerate}

\end{document}
