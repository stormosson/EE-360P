\documentclass[11pt]{article}
\usepackage[utf8]{inputenc}
\usepackage[T1]{fontenc}
\usepackage{amsmath}
\usepackage{amssymb}
\usepackage{amsthm}
\usepackage{mathtools}
\usepackage{fixltx2e}
\usepackage{longtable}
\usepackage{float}
\usepackage{array}
\usepackage{multirow}
\usepackage{multicol}
\usepackage{wrapfig}
\usepackage{rotating}
\usepackage[normalem]{ulem}
\usepackage{textcomp}
\usepackage{marvosym}
\usepackage{wasysym}
\usepackage{hyperref}
\tolerance=1000
\usepackage{caption}
\usepackage{subcaption}
\usepackage[margin=0.75in]{geometry}
\usepackage{enumerate}
\usepackage{showexpl}
\usepackage{color} %red, green, blue, yellow, cyan, magenta, black, white
\definecolor{mygreen}{RGB}{28,172,0} % color values Red, Green, Blue
\definecolor{mylilas}{RGB}{170,55,241}
\newcommand{\todo}{{\LARGE \emph{\color{red}TODO}}}
\hypersetup{
  pdfkeywords={},
  pdfsubject={}}
\newcounter{problem}
\newcounter{solution}
\usepackage{beramono}
\usepackage{listings}
\usepackage[usenames,dvipsnames]{xcolor}
\usepackage[T1]{fontenc}
\usepackage{inconsolata}
\usepackage{color}
\definecolor{pblue}{rgb}{0.13,0.13,1}
\definecolor{pgreen}{rgb}{0,0.5,0}
\definecolor{pred}{rgb}{0.9,0,0}
\definecolor{pgrey}{rgb}{0.46,0.45,0.48}
\lstset{language=Java,
  showspaces=false,
  showtabs=false,
  breaklines=true,
  showstringspaces=false,
  breakatwhitespace=true,
  commentstyle=\color{pgreen},
  keywordstyle=\color{pblue},
  stringstyle=\color{pred},
  basicstyle=\ttfamily,
  moredelim=[il][\textcolor{pgrey}]{$$},
  moredelim=[is][\textcolor{pgrey}]{\%\%}{\%\%}
}
% \parindent 0in
% \parskip 1em.
\begin{document}
\title{EE 360P:\@Concurrent and Distributed Systems \\ Assignment 3}
\author{Student: Eric Crosson (email: eric.s.crosson@utexas.edu)\\
  Student: William ``Stormy'' Mauldin (email: stormymauldin@utexas.edu)}
\date{\today}
\maketitle

This homework contains a programming part (Q1) and a theory part (Q3). The
theory part should be written or typed on a paper and submitted at the beginning
of the class. The source code (Java files) of the programming part must be
uploaded through the canvas before the end of the due date (i.e., 11:59pm in
March 1. The assignment should be done in teams of two. You
should use the templates downloaded from the course github
(https://github.com/vijaygarg1/EE-360P.git). You should not change the file
names and function signatures. In addition, you should not use package for
encapsulation. Note that, we provide some test cases for your convenience in
SimpleTest class. You do not need to modify the class. Please zip and name the
source code as [EID1 EID2].zip. \\

\begin{enumerate}
\item \textbf{(10 points)} 
\begin{enumerate}[(a)]
\item Prove or disprove that every symmetric and transitive relation is reflexive.  
\begin{proof}[Disproof]
Given a set $A = \{a,b,c\}$, let a relation $R$ on $A$ be defined as such: 

\begin{equation*}
R=\{(a,b),(b,a),(a,a),(b,b)\}
\end{equation*}

From the definition of symmetry, it can be seen that $R$ is a symmetric relation since $xRy \Rightarrow yRx$ for some arbitrary $x,y,z\in A$. From the definition of transitivity, it can be seen that $R$ is a transitive relation since $xRy \wedge yRz \Rightarrow xRz$ for some arbitrary $x,y,z\in A$. The definition of reflexivity requires that $\forall x\in A$, $xRx$. However, we can see from the set $A$ definition of the relation $R$ that $c\in A \wedge (c,c)\notin R$, which implies that $R$ is not reflexive. Since $R$ is symmetric and transitive but not reflexive, then the claim that every symmetric and transitive relation is not true.
\end{proof}

\item Prove or disprove that every irreflexive and transitive relation is asymmetric.  
\begin{proof}
Assume that some relation $R$ on a set $A$ is irreflexive and transitive. Assume also that $(x,y)\in R$ for some arbitrary $x,y\in A$ such that $x \neq y$. By the definition of asymmetry, $(y,x)$ cannot be a member of $R$. Now assume that such is the case. Since $R$ was assumed to be transitive, then $(x,y)\in R\wedge(y,x)\in R \Rightarrow (x,x)\in R$, which contradicts the assumption of irreflexivity. Therefore, it must be true that every irreflexive and transitive relation is asymmetric.
\end{proof}
\end{enumerate}
\end{enumerate}
\end{document}
