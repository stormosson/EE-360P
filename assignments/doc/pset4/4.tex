\documentclass[11pt]{article}
\usepackage[utf8]{inputenc}
\usepackage[T1]{fontenc}
\usepackage{amsmath}
\usepackage{amssymb}
\usepackage{amsthm}
\usepackage{mathtools}
\usepackage{fixltx2e}
\usepackage{longtable}
\usepackage{float}
\usepackage{array}
\usepackage{multirow}
\usepackage{multicol}
\usepackage{wrapfig}
\usepackage{rotating}
\usepackage[normalem]{ulem}
\usepackage{textcomp}
\usepackage{marvosym}
\usepackage{wasysym}
\usepackage{hyperref}
\tolerance=1000
\usepackage{caption}
\usepackage{subcaption}
\usepackage[margin=0.75in]{geometry}
\usepackage{enumerate}
\usepackage{showexpl}
\usepackage{color} %red, green, blue, yellow, cyan, magenta, black, white
\definecolor{mygreen}{RGB}{28,172,0} % color values Red, Green, Blue
\definecolor{mylilas}{RGB}{170,55,241}
\newcommand{\todo}{{\LARGE \emph{\color{red}TODO}}}
\hypersetup{
  pdfkeywords={},
  pdfsubject={}}
\newcounter{problem}
\newcounter{solution}
\usepackage{beramono}
\usepackage{listings}
\usepackage[usenames,dvipsnames]{xcolor}
\usepackage[T1]{fontenc}
\usepackage{inconsolata}
\usepackage{color}
\definecolor{pblue}{rgb}{0.13,0.13,1}
\definecolor{pgreen}{rgb}{0,0.5,0}
\definecolor{pred}{rgb}{0.9,0,0}
\definecolor{pgrey}{rgb}{0.46,0.45,0.48}
\lstset{language=Java,
  showspaces=false,
  showtabs=false,
  breaklines=true,
  showstringspaces=false,
  breakatwhitespace=true,
  commentstyle=\color{pgreen},
  keywordstyle=\color{pblue},
  stringstyle=\color{pred},
  basicstyle=\ttfamily,
  moredelim=[il][\textcolor{pgrey}]{$$},
  moredelim=[is][\textcolor{pgrey}]{\%\%}{\%\%}
}
% \parindent 0in
% \parskip 1em.
\begin{document}
\title{EE 360P:\@Concurrent and Distributed Systems \\ Assignment 4}
\author{Student: Eric Crosson (email: eric.s.crosson@utexas.edu)\\
  Student: William ``Stormy'' Mauldin (email: stormymauldin@utexas.edu)}
\date{\today}
\maketitle

\begin{enumerate}
\item \textbf{(20 points)} 
\begin{enumerate}[(a)]
\item Extend Lamport's mutex algorithm to solve $k$-mutual exclusion problem which allows at most $k$ processes to be in the critical section concurrently.

\begin{itemize}
\item \italicize{Notes}: Each process has a counter, initialized to the number of processes.
\item To request the critical section, a process sends a timestamped message to all other processes and adds a timestamped request to the queue.
\item On receiving a request message, the request and its timestamp are stored in the queue and a timestamped acknowledgement is sent back.
\item To release the critical section, a process' counter is incremented and it sends a release message to all other processes.
\item On receiving a release message, the corresponding request is deleted from the queue and increment their counter. NOTE: I AM AWARE OF THE AMBIGUITY OF "COUNTER" HERE, BUT HAVE NOT MODIFIED YET
\item A process determines that it can access the critical section if and only if all of the following are true: 
\begin{itemize}
\item It has a request in the queue with timestamp $t$
\item $t$ is less than all other requests in the queue
\item It has received a message from every other process with timestamp greater than $t$.
\end{itemize}
Immediately before entering the critical section, the process decrements its counter.
\end{itemize}


\item Extend Ricart and Agrawala's mutex algorithm to solve the $k$-mutual exclusion problem.   

General idea: condition on entering CS: $k - count \leq t \leq (k - count) + num$

\end{enumerate}
\end{enumerate}
\end{document}
